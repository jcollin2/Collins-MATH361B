\documentclass{article}
\usepackage{JBpack}
\usepackage{fancyhdr, fancybox}
\usepackage{extramarks, multicol} % Required for headers and footers

\def\prog#1{
\vspace{.1in}\begin{mdframed} \begin{center} \textbf{Programming Reminders} \end{center}#1 \end{mdframed} }

\topmargin=-0.45in
\evensidemargin=0in
\oddsidemargin=0in
\textwidth=7.5in
\textheight=9.0in
\headsep=0.25in 
\headheight = 20pt
\hoffset = -.5in




\pagestyle{fancy}

\lhead{\large Due Date: \duedate}
%\chead{}
\rhead{\large Problem \problemnumber: \probname}
\renewcommand\headrulewidth{0.4pt} % Size of the header rule
\renewcommand{\labelitemii}{$\star$}







%%%%%% Assignment Data %%%%%%%%%%
\newcommand\problemnumber{I.3}
\newcommand\duedate{}
\newcommand\assignmenttype{Intro To Programming}
\newcommand\foldername{\tt{IntroToProgramming} }
\newcommand\filename{\tt{PartialSum\_LastName.py}\;\;}
\newcommand\probname{Partial Sums}







\begin{document}
\begin{multicols}{2}




%%%%%%% Title %%%%%%%%%%	
\begin{minipage}{.5\textwidth}
	 \begin{center} \shadowbox{\begin{Bcenter} \Huge \underline{Problem \problemnumber} \vspace{.25in} \\  \Huge \textbf{\probname}  \end{Bcenter}  }
	 
	 {\Large Due Date: \duedate
	 
	 Folder: \foldername
	 
	 File Name: \filename}
	  \end{center}
 \end{minipage}

\columnbreak


%%%%% Learning Objectives %%%%%%%
\begin{minipage}{.45\textwidth}
	\begin{center}
	
	\textbf{\Large Learning Objectives}
	
	\doublebox{
		\begin{Bitemize}
			\item Programming Skills
			\begin{Bitemize}
				\item For loops
			\end{Bitemize}
			\item Importing packages
			\item Using numpy arrays
		\end{Bitemize}
	} 
	\end{center}
\end{minipage}
\end{multicols}






%%%%%%% Background %%%%%%%%%%%%
\section*{Problem Background}
	Recall from Calculus 2 the idea of a sequence of partial sums,
	\[ s_n = \sum_{i=0}^n a_i.\]
	This was used to discuss series convergence where we said if the sequence $s_n$ converged, then the series $ \sum_{i=0}^\infty a_i$ also converged.  For this problem, we will compute the first $N$ terms of the \emph{sequence} of partial sums, and use these terms to guess as to whether the associated series converges or diverges.  However, in order to get a good idea of the behavior of the series, we will use a large value for $N$.  This massive computation can be simplified with the use of loops.
	
	In addition, we will be storing the the terms of the sequence in a numpy array.  There are some subtle differences between numpy arrays and python lists.  Numpy arrays allow the use of linear algebra functions, which can be useful in many situations.
	\prog{
		\begin{itemize}
			\item Syntax for a \tt{for} loop: \tt{for ii in range(N):}
			\item Import \tt{numpy} with \tt{import numpy as np}
			\item Create a numpy array of zeros with \tt{x = np.zeros((10))}
		\end{itemize}
		}




%%%%%%%%%% Program Criteria %%%%%%%%%
\section*{Program Criteria}
	Write a program that does the following:
	\begin{itemize}
		\item Create an input variable \tt{N} for the total number of terms in the partial sum sequence.
		\item Generate the first $N$ terms for the following partial sum sequences using a \tt{for} loop, not a built-in function,
		\begin{itemize}
			\item $\ds s_n = \sum_{i=1}^n \frac{\ln(i^4 +i+3)}{\sqrt{i}+3}$
			\item $\ds t_n = \sum_{i=1}^n \frac{e^{i/100}}{i^{10}}$
			\item A partial sum of your creation
		\end{itemize}
		\item Print out the first 15 terms and the last 15 terms in each sequence, with an appropriate description.
	\end{itemize}







%%%%%%%%% Deliverables %%%%%%%%%
\section*{Deliverables}
	Place the following in a folder named \foldername in your repository:
	\begin{itemize}
		\item A Python file \filename  that satisfies the program criteria.
		\item A Latex document \tt{PartialSum\_LastName.pdf} with the following information:
		\begin{itemize}
			\item Write down the partial sum terms that you created
			\item State whether you think each series will converge or diverge and explain why.
			\item If you think the series will converge, give an estimate of what it will converge to.
			\item How many terms did you use to come to this conclusion?  Why did you use that many?
		\end{itemize}
	\end{itemize}

	
\end{document}

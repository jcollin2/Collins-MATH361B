\documentclass{article}
\usepackage{JBpack}
\usepackage{fancyhdr, fancybox}
\usepackage{extramarks, multicol} % Required for headers and footers
\usepackage{sectsty}

\def\prog#1{
\vspace{.1in}\begin{mdframed} \begin{center} \textbf{Programming Reminders} \end{center}#1 \end{mdframed} }



\sectionfont{\underline}




\topmargin=-0.45in
\evensidemargin=0in
\oddsidemargin=0in
\textwidth=7.5in
\textheight=9.0in
\headsep=0.25in 
\headheight = 20pt
\hoffset = -.5in




\pagestyle{fancy}

\lhead{\large Due Date: \duedate}
%\chead{}
\rhead{\large Problem \problemnumber: \probname}
\renewcommand\headrulewidth{0.4pt} % Size of the header rule
\renewcommand{\labelitemii}{$\star$}







%%%%%% Assignment Data %%%%%%%%%%
\newcommand\problemnumber{I.6}
\newcommand\duedate{2/15/2019}
\newcommand\assignmenttype{Intro To Programming}
\newcommand\foldername{\tt{IntroToProgramming} }
\newcommand\filename{\tt{FibSeq2\_LastName.py}\;\;}
\newcommand\probname{Fibonacci Sequence 2}







\begin{document}
\begin{multicols}{2}




%%%%%%% Title %%%%%%%%%%	
\begin{minipage}{.5\textwidth}
	 \begin{center} \shadowbox{\begin{Bcenter} \Huge \underline{Problem \problemnumber} \vspace{.25in} \\  \Huge \textbf{\probname}  \end{Bcenter}  }
	 
	 {\Large Due Date: \duedate
	 
	 Folder: \foldername
	 
	 File Name: \filename}
	  \end{center}
 \end{minipage}

\columnbreak


%%%%% Learning Objectives %%%%%%%
\begin{minipage}{.45\textwidth}
	\begin{center}
	
	\textbf{\Large Learning Objectives}
	
	\doublebox{
		\begin{Bitemize}
			\item Programming Skills
			\begin{Bitemize}
				\item For loops
				\item Conditionals
				\item Mod function
			\end{Bitemize}
			\item Conjecture a pattern from data
		\end{Bitemize}
	} 
	\end{center}
\end{minipage}
\end{multicols}






%%%%%%% Background %%%%%%%%%%%%
\section*{Problem Background}
	Recall the Fibonacci sequence $F_0 = 0, F_1 = 1$,
	\[ F_{n} = F_{n-1} + F_{n-2}.\]
	We will be considering this sequence and looking at which terms are divisible by certain integers.  For instance, how many terms are even, how many are multiples of 3, and so on.
	
	
	
	
	
	
	
	
	
	\prog{
		\begin{itemize}
			\item Syntax for a \tt{for} loop: \tt{for ii in range(N):}
			\item To create an empty list use \tt{my\_list = []}
			\item To check if \tt{n} is a multiple of \tt{m}, use \tt{if(n \% m == 0)}
		\end{itemize}
		}




%%%%%%%%%% Program Criteria %%%%%%%%%
\section*{Program Criteria}
	Write a program that does the following:
	\begin{itemize}
		\item Create an input variable \tt{N} for the number of terms to generate in your sequence.
		\item Create an input variable \tt{m} to hold the multiple you will be checking.
		\item Generate the first $N$ terms of the Fibonacci sequence, using the typical initial terms, described above and store them in a list.
		\item In a different list, store all terms of the Fibonacci sequence that are multiple of \tt{m}.
		\item Print out the number of terms that are divisible by \tt{m}.
	\end{itemize}







%%%%%%%%% Deliverables %%%%%%%%%
\section*{Deliverables}
	An interesting question to ask is which elements of the Fibonacci sequence are even?  Which are multiples of 3?  Multiples of 4?  Prime?  These are questions that can be at least conjectured at with the code in the Problem.
	
	Place the following in a folder named \foldername in your repository:
	\begin{itemize}
		\item A Python file \filename  that satisfies the program criteria.
		\item A Latex document \tt{FibSeq2\_LastName.pdf} with the following information:
		\begin{itemize}
			\item List all multiple of 4 in the first 50 Fibonacci numbers.
			\item What percentage of the first 10,000 Fibonacci numbers are even?
			\item Can you conjecture a pattern for the percentage of Fibonacci numbers that are multiples of $m$.  (For instance, one conjecture would be that the percentage of Fibonacci numbers that are multiple of $m$ is $\frac{3}{m^2}$.)
		\end{itemize}
	\end{itemize}

	
\end{document}

\documentclass{article}
\usepackage{JBpack}
\usepackage{fancyhdr, fancybox}
\usepackage{extramarks, multicol} % Required for headers and footers
\usepackage{sectsty, hyperref}

\def\prog#1{
\vspace{.1in}\begin{mdframed} \begin{center} \textbf{Programming Reminders} \end{center}#1 \end{mdframed} }



\sectionfont{\underline}




\topmargin=-0.45in
\evensidemargin=0in
\oddsidemargin=0in
\textwidth=7.5in
\textheight=9.0in
\headsep=0.25in 
\headheight = 20pt
\hoffset = -.5in




\pagestyle{fancy}

\lhead{\large Due Date: \duedate}
%\chead{}
\rhead{\large Problem \problemnumber: \probname}
\renewcommand\headrulewidth{0.4pt} % Size of the header rule
\renewcommand{\labelitemii}{$\star$}







%%%%%% Assignment Data %%%%%%%%%%
\newcommand\problemnumber{F.1}
\newcommand\duedate{5/3/2019}
\newcommand\assignmenttype{Final Project}
\newcommand\foldername{\tt{FinalProject} }
\newcommand\filename{\tt{F1\_Prob265\_Name.py}\;\;}
\newcommand\probname{Problem 265: Binary Circles}
\newcommand\probnametwo{}







\begin{document}
\begin{multicols}{1}




%%%%%%% Title %%%%%%%%%%	
\begin{minipage}{\textwidth}
	 \begin{center} \shadowbox{\begin{Bcenter} \Huge \underline{Problem \problemnumber} \vspace{.25in} \\  \Huge \textbf{\probname} \\ \Huge \textbf{\probnametwo}  \end{Bcenter}  }
	 
	 {\Large Due Date: \duedate
	 
	 Folder: \foldername
	 
	 File Name: \filename
	 
	 Points: 20 points}
	  \end{center}
 \end{minipage}

\columnbreak


%%%%% Learning Objectives %%%%%%%
%\begin{minipage}{.45\textwidth}
%	\begin{center}
%	
%	\textbf{\Large Learning Objectives}
%	
%	\doublebox{
%		\begin{Bitemize}
%			\item Consider properties of quadratic residues
%		\end{Bitemize}
%	} 
%	\end{center}
%\end{minipage}
\end{multicols}






%%%%%%% Background %%%%%%%%%%%%
\section*{Problem Background}
	Assume we have $2^N$ binary digits, that is 0's or 1's (henceforth called \emph{bits}).  These bits can be arranged in a circle in many different ways.  For a visual of this, see {\color{blue}\href{https://projecteuler.net/problem=265}{Problem 265}}.  These are called \emph{binary circles}.  At this link, an example of $N=3$ is given.  If we examine all the $N$ element subsequences of the binary circles, taking each subsequence in clockwise order, we want to determine which circles exist such that all $2^N$ subsequences are distinct.
	
	Obviously, once you find such a circle, you can simply rotate the elements and get another circle.  In light of that, we will ignore all rotations of a circle.  When  $N=3$, ignoring circle rotations, there are only 2 such binary circles.
	
	
 
 	
 	
 	
 	
 	
 	
 	

	
	
	
	
	
	
	
	
%	\prog{
%		\begin{itemize}
%			\item Create function: \tt{def func\_name(params):}
%			\item Loops syntax: \tt{for ii in range(N)}
%			\item Create empty list: \tt{my\_list = []}
%			\item Append to list: \tt{my\_list.append(new\_thing)}
%		\end{itemize}
%		}




%%%%%%%%%% Program Criteria %%%%%%%%%
\section*{Program Criteria}
	Write a program that does the following:
	\begin{itemize}
		\item Has an input variable \tt{N} to determine the $2^N$ number of bits on the circle. 
		\item Determine how many binary circles have all distinct $N$ length subsequences.  Be sure to ignore rotations of the circle. (That is, all $2^N$ rotations of a circle with distinct $N$ length subsequences will count as just one such binary circle.)
		\item Print out the number of binary circles, as well as the elements of the circles, starting at the top and going in clockwise order.
	\end{itemize}







%%%%%%%%% Deliverables %%%%%%%%%
\section*{Deliverables}
	
	
	Place the following in a folder named \foldername in your repository:
	\begin{itemize}
		\item A Python file \filename  that satisfies the program criteria.
		\item A PDF file \tt{F1\_Prob265\_Name.pdf} that describes how your program works.  This should be a description of how you went about solving this problem.  You should go into some detail about your solution method, but I don't want to see something about every \tt{if} statement and \tt{for} loop.
	\end{itemize}

	
\end{document}

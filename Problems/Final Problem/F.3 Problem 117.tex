\documentclass{article}
\usepackage{JBpack}
\usepackage{fancyhdr, fancybox}
\usepackage{extramarks, multicol} % Required for headers and footers
\usepackage{sectsty, hyperref}

\def\prog#1{
\vspace{.1in}\begin{mdframed} \begin{center} \textbf{Programming Reminders} \end{center}#1 \end{mdframed} }



\sectionfont{\underline}




\topmargin=-0.45in
\evensidemargin=0in
\oddsidemargin=0in
\textwidth=7.5in
\textheight=9.0in
\headsep=0.25in 
\headheight = 20pt
\hoffset = -.5in




\pagestyle{fancy}

\lhead{\large Due Date: \duedate}
%\chead{}
\rhead{\large Problem \problemnumber: \probname}
\renewcommand\headrulewidth{0.4pt} % Size of the header rule
\renewcommand{\labelitemii}{$\star$}







%%%%%% Assignment Data %%%%%%%%%%
\newcommand\problemnumber{F.3}
\newcommand\duedate{5/3/2019}
\newcommand\assignmenttype{Final Project}
\newcommand\foldername{\tt{FinalProject} }
\newcommand\filename{\tt{F3\_Prob116\_Name.py}\;\;}
\newcommand\probname{Problem 116}
\newcommand\probnametwo{}







\begin{document}
\begin{multicols}{1}




%%%%%%% Title %%%%%%%%%%	
\begin{minipage}{\textwidth}
	 \begin{center} \shadowbox{\begin{Bcenter} \Huge \underline{Problem \problemnumber} \vspace{.25in} \\  \Huge \textbf{\probname} \\ \Huge \textbf{\probnametwo}  \end{Bcenter}  }
	 
	 {\Large Due Date: \duedate
	 
	 Folder: \foldername
	 
	 File Name: \filename
	 
	 Points: 20 points}
	  \end{center}
 \end{minipage}

\columnbreak


%%%%% Learning Objectives %%%%%%%
%\begin{minipage}{.45\textwidth}
%	\begin{center}
%	
%	\textbf{\Large Learning Objectives}
%	
%	\doublebox{
%		\begin{Bitemize}
%			\item Consider properties of quadratic residues
%		\end{Bitemize}
%	} 
%	\end{center}
%\end{minipage}
\end{multicols}






%%%%%%% Background %%%%%%%%%%%%
\section*{Problem Background}
	Begin by looking at the Project Euler page for this problem, {\color{blue} \href{https://projecteuler.net/problem=116}{Problem 116}}.  We are given $N$ blank squares, and a collection of red tiles that are 2 squares long, green tiles that are 3 squares long, and blue tiles that are 4 squares long.  The goal is to determine how many different ways there are to place the red tiles, green tiles and blue tiles, separately, onto the $N$ blank squares.  We assume the colors cannot be mixed for this problem.  When there are 5 blank squares, the solution is shown on the Project Euler page.


	
	
 
 	
 	
 	
 	
 	
 	
 	

	
	
	
	
	
	
	
	
%	\prog{
%		\begin{itemize}
%			\item Create function: \tt{def func\_name(params):}
%			\item Loops syntax: \tt{for ii in range(N)}
%			\item Create empty list: \tt{my\_list = []}
%			\item Append to list: \tt{my\_list.append(new\_thing)}
%		\end{itemize}
%		}




%%%%%%%%%% Program Criteria %%%%%%%%%
\section*{Program Criteria}
	Write a program that does the following:
	\begin{itemize}
		\item
	\end{itemize}







%%%%%%%%% Deliverables %%%%%%%%%
\section*{Deliverables}
	
	
	Place the following in a folder named \foldername in your repository:
	\begin{itemize}
		\item A Python file \filename  that satisfies the program criteria.
	\end{itemize}

	
\end{document}

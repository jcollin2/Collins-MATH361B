\documentclass{article}
\usepackage{JBpack}
\usepackage{fancyhdr, fancybox}
\usepackage{extramarks, multicol} % Required for headers and footers
\usepackage{sectsty, hyperref}

\def\prog#1{
\vspace{.1in}\begin{mdframed} \begin{center} \textbf{Programming Reminders} \end{center}#1 \end{mdframed} }



\sectionfont{\underline}




\topmargin=-0.45in
\evensidemargin=0in
\oddsidemargin=0in
\textwidth=7.5in
\textheight=9.0in
\headsep=0.25in 
\headheight = 20pt
\hoffset = -.5in




\pagestyle{fancy}

\lhead{\large Due Date: \duedate}
%\chead{}
\rhead{\large Problem \problemnumber: \probname}
\renewcommand\headrulewidth{0.4pt} % Size of the header rule
\renewcommand{\labelitemii}{$\star$}







%%%%%% Assignment Data %%%%%%%%%%
\newcommand\problemnumber{F.3}
\newcommand\duedate{5/3/2019}
\newcommand\assignmenttype{Final Project}
\newcommand\foldername{\tt{FinalProject} }
\newcommand\filename{\tt{F3\_Prob116\_Name.py}\;\;}
\newcommand\probname{Problem 116}
\newcommand\probnametwo{}







\begin{document}
\begin{multicols}{1}




%%%%%%% Title %%%%%%%%%%	
\begin{minipage}{\textwidth}
	 \begin{center} \shadowbox{\begin{Bcenter} \Huge \underline{Problem \problemnumber} \vspace{.25in} \\  \Huge \textbf{\probname} \\ \Huge \textbf{\probnametwo}  \end{Bcenter}  }
	 
	 {\Large Due Date: \duedate
	 
	 Folder: \foldername
	 
	 File Name: \filename
	 
	 Points: 20 points}
	  \end{center}
 \end{minipage}

\columnbreak


%%%%% Learning Objectives %%%%%%%
%\begin{minipage}{.45\textwidth}
%	\begin{center}
%	
%	\textbf{\Large Learning Objectives}
%	
%	\doublebox{
%		\begin{Bitemize}
%			\item Consider properties of quadratic residues
%		\end{Bitemize}
%	} 
%	\end{center}
%\end{minipage}
\end{multicols}






%%%%%%% Background %%%%%%%%%%%%
\section*{Problem Background}
	Begin by looking at the Project Euler page for this problem, {\color{blue} \href{https://projecteuler.net/problem=116}{Problem 116}}.  We are given $N$ blank squares, and a collection of red tiles that are 2 squares long, green tiles that are 3 squares long, and blue tiles that are 4 squares long.  The goal is to determine how many different ways there are to place the red tiles, green tiles and blue tiles, separately, onto the $N$ blank squares.  We assume the colors cannot be mixed for this problem.  When there are 5 blank squares, the solution is shown on the Project Euler page.


	
	
 
 	
 	
 	
 	
 	
 	
 	

	
	
	
	
	
	
	
	
%	\prog{
%		\begin{itemize}
%			\item Create function: \tt{def func\_name(params):}
%			\item Loops syntax: \tt{for ii in range(N)}
%			\item Create empty list: \tt{my\_list = []}
%			\item Append to list: \tt{my\_list.append(new\_thing)}
%		\end{itemize}
%		}




%%%%%%%%%% Program Criteria %%%%%%%%%
\section*{Program Criteria}
	Write a program that does the following:
	\begin{itemize}
		\item Create an input variable \tt{N} for the number of spaces you will be filling with theses tiles.
		\item Compute the number of different ways you can fill those $N$ spaces with the red, blue and green tiles separately.
		\item Print out, with appropriate text, the total number of ways to fill the spaces, with each color tile.
		\item (Extra Credit) Do problem 117, that is compute the number of ways to fill the spaces if you can mix the colored tiles, for instance, using both green and blue to cover the spaces.
	\end{itemize}







%%%%%%%%% Deliverables %%%%%%%%%
\section*{Deliverables}
	
	
	Place the following in a folder named \foldername in your repository:
	\begin{itemize}
		\item A Python file \filename  that satisfies the program criteria.
		\item A PDF file \tt{F3\_Prob116\_Name.pdf} that describes how your program works.  This should be a description of how you went about solving this problem.  You should go into some detail about your solution method, but I don't want to see something about every \tt{if} statement and \tt{for} loop.  As an example of the type of description I'm looking for, see the file \tt{Goldbach\_explanation.doc} in the \tt{Final Problem} folder of my repo.
		\item In the same PDF, discuss any pattern you see as you change $N$.  Examples could be how the number of ways to fill the spaces with all colors changes as $N$ grows.  Another could be how the number of ways to fill it with red tiles compares to the number of ways with green tiles, and similarly with blue.  Other questions can be asked as well, discuss any others that might come to mind.

	\end{itemize}

	
\end{document}

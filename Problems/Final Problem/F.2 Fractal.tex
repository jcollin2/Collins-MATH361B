\documentclass{article}
\usepackage{JBpack}
\usepackage{fancyhdr, fancybox}
\usepackage{extramarks, multicol} % Required for headers and footers
\usepackage{sectsty, hyperref}

\def\prog#1{
\vspace{.1in}\begin{mdframed} \begin{center} \textbf{Programming Reminders} \end{center}#1 \end{mdframed} }



\sectionfont{\underline}




\topmargin=-0.45in
\evensidemargin=0in
\oddsidemargin=0in
\textwidth=7.5in
\textheight=9.0in
\headsep=0.25in 
\headheight = 20pt
\hoffset = -.5in




\pagestyle{fancy}

\lhead{\large Due Date: \duedate}
%\chead{}
\rhead{\large Problem \problemnumber: \probname}
\renewcommand\headrulewidth{0.4pt} % Size of the header rule
\renewcommand{\labelitemii}{$\star$}







%%%%%% Assignment Data %%%%%%%%%%
\newcommand\problemnumber{F.2}
\newcommand\duedate{5/3/2019}
\newcommand\assignmenttype{Final Project}
\newcommand\foldername{\tt{FinalProject} }
\newcommand\filename{\tt{F2\_Fractal\_Name.py}\;\;}
\newcommand\probname{Fractals}
\newcommand\probnametwo{}







\begin{document}
\begin{multicols}{1}




%%%%%%% Title %%%%%%%%%%	
\begin{minipage}{\textwidth}
	 \begin{center} \shadowbox{\begin{Bcenter} \Huge \underline{Problem \problemnumber} \vspace{.25in} \\  \Huge \textbf{\probname} \\ \Huge \textbf{\probnametwo}  \end{Bcenter}  }
	 
	 {\Large Due Date: \duedate
	 
	 Folder: \foldername
	 
	 File Name: \filename
	 
	 Points: 20 points}
	  \end{center}
 \end{minipage}

\columnbreak


%%%%% Learning Objectives %%%%%%%
%\begin{minipage}{.45\textwidth}
%	\begin{center}
%	
%	\textbf{\Large Learning Objectives}
%	
%	\doublebox{
%		\begin{Bitemize}
%			\item Consider properties of quadratic residues
%		\end{Bitemize}
%	} 
%	\end{center}
%\end{minipage}
\end{multicols}






%%%%%%% Background %%%%%%%%%%%%
\section*{Problem Background}
	Start with a shape, say a triangle.  Consider each point on the triangle to be defined by a set of vectors 
	\[ S_0 = \{ \begin{bmatrix} x \\ y \end{bmatrix} : \text{the point } (x,y) \text{ is a point on the triangle} \}.\]
	
	  Different values for $t$ correspond to different places on the triangle.  To generate a fractal, we will perform 3 transformations on all of these vectors to generate a new set of points $S_1$.  Here is one example of these transformations.  For each vector $v \in S_0$, we generate 3 new vectors $w_1, w_2, w_3$ 
	
	\begin{align*}
		w_1 &= \begin{bmatrix}
			\frac{1}{4} & 0 \\ 0 & \frac{1}{4} \end{bmatrix}v + \begin{bmatrix} 0 \\ 0 \end{bmatrix} \\
			w_2 &= \begin{bmatrix}\frac{1}{4} & 0 \\ 0 & \frac{1}{4}  \end{bmatrix}v + \begin{bmatrix} 0.5 \\ 0 \end{bmatrix} \\
			w_3 &= \begin{bmatrix} \frac{1}{4} & 0 \\ 0 & \frac{1}{4} \end{bmatrix}v + \begin{bmatrix} 0.5 \\ 0.5 \end{bmatrix}
	\end{align*}
	
	All three of these new vectors are then added to a new set $S_1$.  If this process is performed for all vectors $v \in S_0$, we generate a new set $S_1$.  When all the points in $S_1$ are plotted we see three new triangles.  We then perform this transformation yet again on all vectors in $S_1$ to generate a set $S_2$.  By continuing this process repeatedly, we obtain a fractal image.
	
	Your project will be to generate all the $S_i$ sets up to some maximum value for $i$.  The plot each one to see how the image changes.  In general, the transformations shown above can be written as
	\begin{align*}
		w_1 &= A_1v + b_1 \\
		w_2 &= A_2v + b_2 \\
		w_3 &= A_3v + b_3 	
	\end{align*}
	where $A_i$ are matrices and $b_i$ are vectors.  You will also be changing the matrices and vectors to see how it affects the fractal image.  In addition, you will change the initial image to see how it affects the resulting fractal.


	
	
 
 	
 	
 	
 	
 	
 	
 	

	
	
	
	
	
	
	
	
%	\prog{
%		\begin{itemize}
%			\item Create function: \tt{def func\_name(params):}
%			\item Loops syntax: \tt{for ii in range(N)}
%			\item Create empty list: \tt{my\_list = []}
%			\item Append to list: \tt{my\_list.append(new\_thing)}
%		\end{itemize}
%		}




%%%%%%%%%% Program Criteria %%%%%%%%%
\section*{Program Criteria}
	Write a program that does the following:
	\begin{itemize}
		\item Create an initial image.  This could be a geometric object, your name, a face.  Really anything that can be plotted in Python and described using vectors, that is points $(x,y)$.
		\item Apply the above transformations, or something similar, to your image, thus obtaining a new set of  points that looks like 3 new transformed copies of your image.
		\item Continue applying the transformations to the  previous points for $N$ iterations.  Be sure to make $N$ a variable so that the number of iterations can be changed easily.  
		\item Plot the final image that is created after $N$ iterations.
	\end{itemize}







%%%%%%%%% Deliverables %%%%%%%%%
\section*{Deliverables}
	
	
	Place the following in a folder named \foldername in your repository:
	\begin{itemize}
		\item A Python file \filename  that satisfies the program criteria.
		\item A PDF file \tt{F1\_Prob265\_Name.pdf} that describes how your program works.  This should be a description of how you went about solving this problem.  You should go into some detail about your solution method, but I don't want to see something about every \tt{if} statement and \tt{for} loop.  As an example of the type of description I'm looking for, see the file \tt{Goldbach\_explanation.doc} in the \tt{Final Problem} folder of my repo.
		\item Include in your PDF file the original image you used.  
		\item Include also at least one of the following:
		\begin{itemize}
			\item Use a different starting image.  Discuss how the final image after $N$ iterations compares to what you got with your first image.
			\item Change the transformation done to the image.  This could involve changing the matrix $A_i$, changing the vectors $b_i$, or adding another transformation entirely.  Discuss how the final image compares to the final image using your original transformations.
		\end{itemize}
	\end{itemize}

	
\end{document}

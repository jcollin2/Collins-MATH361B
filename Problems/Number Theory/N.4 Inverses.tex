\documentclass{article}
\usepackage{JBpack}
\usepackage{fancyhdr, fancybox}
\usepackage{extramarks, multicol} % Required for headers and footers
\usepackage{sectsty}

\def\prog#1{
\vspace{.1in}\begin{mdframed} \begin{center} \textbf{Programming Reminders} \end{center}#1 \end{mdframed} }



\sectionfont{\underline}




\topmargin=-0.45in
\evensidemargin=0in
\oddsidemargin=0in
\textwidth=7.5in
\textheight=9.0in
\headsep=0.25in 
\headheight = 20pt
\hoffset = -.5in




\pagestyle{fancy}

\lhead{\large Due Date: \duedate}
%\chead{}
\rhead{\large Problem \problemnumber: \probname}
\renewcommand\headrulewidth{0.4pt} % Size of the header rule
\renewcommand{\labelitemii}{$\star$}







%%%%%% Assignment Data %%%%%%%%%%
\newcommand\problemnumber{N.4}
\newcommand\duedate{3/25/2019}
\newcommand\assignmenttype{Number Theory}
\newcommand\foldername{\tt{NumberTheory} }
\newcommand\filename{\tt{N4\_Inverses\_Name.py}\;\;}
\newcommand\probname{Inverses of $\Zbb_m$}
\newcommand\probnametwo{}







\begin{document}
\begin{multicols}{2}




%%%%%%% Title %%%%%%%%%%	
\begin{minipage}{.5\textwidth}
	 \begin{center} \shadowbox{\begin{Bcenter} \Huge \underline{Problem \problemnumber} \vspace{.25in} \\  \Huge \textbf{\probname} \\ \Huge \textbf{\probnametwo}  \end{Bcenter}  }
	 
	 {\Large Due Date: \duedate
	 
	 Folder: \foldername
	 
	 File Name: \filename
	 
	 Points: 2 points}
	  \end{center}
 \end{minipage}

\columnbreak


%%%%% Learning Objectives %%%%%%%
\begin{minipage}{.45\textwidth}
	\begin{center}
	
	\textbf{\Large Learning Objectives}
	
	\doublebox{
		\begin{Bitemize}
			\item Mix of programming skills
		\end{Bitemize}
	} 
	\end{center}
\end{minipage}
\end{multicols}






%%%%%%% Background %%%%%%%%%%%%
\section*{Problem Background}
	Recall the basics of modular arithmetic and the group $\Zbb_m$.  We know that the set of all residue classes is given by,
	\[ \Zbb_m = \{ 0, 1, 2, \ldots, (m-1) \}. \]
	Addition, subtraction and multiplication on these elements is done similarly to regular integers, but using modular arithmetic.  
	
	For example, assume we are considering $\Zbb_7 = \{ 0,1,2,3,4,5,6 \}$.  In $\Zbb_7$ we have that 5+4=2, because $9 \equiv 2 (mod \;7)$.  Also, $2 \cdot 4 = 1$, because $2 \cdot 7 = 8$ and $8 \equiv 1 (mod \;7)$.  
	
	Many things are similar between $\Zbb_m$ and the integers, but some things are different.  One of those things involves multiplicative inverses.  A \textbf{multiplicative  inverse} of an element $a$ is an element $b$ such that $a \cdot b = 1$.  So for instance, about we saw that in $\Zbb_7$, when you multiple 2 and 4, they equal 1.  Therefore, both 2 and 7 are inverses of each other in the set $\Zbb_7$.  The key change in $\Zbb_m$ is that not all element have an inverse.  The number of inverses and which elements have inverses will change depending on what the $m$ is in $\Zbb_m$.    

 
 	
 	
 	
 	
 	
 	
 	

	
	
	
	
	
	
	
	
%	\prog{
%		\begin{itemize}
%			\item Create function: \tt{def func\_name(params):}
%			\item Loops syntax: \tt{for ii in range(N)}
%			\item Create empty list: \tt{my\_list = []}
%			\item Append to list: \tt{my\_list.append(new\_thing)}
%		\end{itemize}
%		}




%%%%%%%%%% Program Criteria %%%%%%%%%
\section*{Program Criteria}
	Write a program that does the following:
	\begin{itemize}
		\item Create an input variable \tt{m} that will represent which $\Zbb_m$ set we are working with.
		\item Determine which elements of $\Zbb_m$ are have a multiplicative inverse.
		\item Print out all the elements of $\Zbb_m$ that have a multiplicative inverse and how many such elements there are, with appropriate descriptive text.	
	\end{itemize}







%%%%%%%%% Deliverables %%%%%%%%%
\section*{Deliverables}
	
	
	Place the following in a folder named \foldername in your repository:
	\begin{itemize}
		\item A Python file \filename  that satisfies the program criteria.
		\item A pdf file \tt{N4\_Inverses\_Name.pdf} describe a simple test for whether a particular element of $\Zbb_m$ has an inverse.  This test will probably depend on $m$.  This should not be a description of your program, but a simpler test that will describe all elements that have an inverse.
	\end{itemize}

	
\end{document}

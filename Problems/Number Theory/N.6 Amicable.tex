\documentclass{article}
\usepackage{JBpack}
\usepackage{fancyhdr, fancybox}
\usepackage{extramarks, multicol} % Required for headers and footers
\usepackage{sectsty}

\def\prog#1{
\vspace{.1in}\begin{mdframed} \begin{center} \textbf{Programming Reminders} \end{center}#1 \end{mdframed} }



\sectionfont{\underline}




\topmargin=-0.45in
\evensidemargin=0in
\oddsidemargin=0in
\textwidth=7.5in
\textheight=9.0in
\headsep=0.25in 
\headheight = 20pt
\hoffset = -.5in




\pagestyle{fancy}

\lhead{\large Due Date: \duedate}
%\chead{}
\rhead{\large Problem \problemnumber: \probname}
\renewcommand\headrulewidth{0.4pt} % Size of the header rule
\renewcommand{\labelitemii}{$\star$}







%%%%%% Assignment Data %%%%%%%%%%
\newcommand\problemnumber{N.6}
\newcommand\duedate{3/29/2019}
\newcommand\assignmenttype{Number Theory}
\newcommand\foldername{\tt{NumberTheory} }
\newcommand\filename{\tt{N6\_Amicable\_Name.py}\;\;}
\newcommand\probname{Amicable Numbers}
\newcommand\probnametwo{}







\begin{document}
\begin{multicols}{2}




%%%%%%% Title %%%%%%%%%%	
\begin{minipage}{.5\textwidth}
	 \begin{center} \shadowbox{\begin{Bcenter} \Huge \underline{Problem \problemnumber} \vspace{.25in} \\  \Huge \textbf{\probname} \\ \Huge \textbf{\probnametwo}  \end{Bcenter}  }
	 
	 {\Large Due Date: \duedate
	 
	 Folder: \foldername
	 
	 File Name: \filename
	 
	 Points: 2 points}
	  \end{center}
 \end{minipage}

\columnbreak


%%%%% Learning Objectives %%%%%%%
\begin{minipage}{.45\textwidth}
	\begin{center}
	
	\textbf{\Large Learning Objectives}
	
	\doublebox{
		\begin{Bitemize}
			\item Use previously defined function
			\item Use temporary variable
		\end{Bitemize}
	} 
	\end{center}
\end{minipage}
\end{multicols}






%%%%%%% Background %%%%%%%%%%%%
\section*{Problem Background}
	Let $d(n)$ be the sum of all proper divisors of $n$.  Recall that the proper divisors of $n$ are all natural numbers strictly less than $n$ that divide $n$.  For instance, if $n = 220$, then the proper divisors of 220 are 1, 2, 4, 5, 10, 11, 20,22,44,55, and 110.  Then $$d(220) = 1 + 2+4+5+10+11+20+22+44+55+110 = 284.$$   
	
	An \textbf{amicable pair} is a pair of natural numbers $a,b$, such that $d(a) = b$ and $d(b) = a$.  Each of $a$ and $b$ are called amicable numbers.  To continue with our example, we examine $d(284)$.  The proper divisors of 284 are 1,2,4,71,and 142.  Then $$d(284) = 1+2+4+71+142 = 220.$$
	 
	Since $d(220) = 284$ and $d(284)=220$ then 220 and 284 are an amicable pair, and each is an amicable number.  
 
 	
 	
 	
 	
 	
 	
 	

	
	
	
	
	
	
	
	
%	\prog{
%		\begin{itemize}
%			\item Create function: \tt{def func\_name(params):}
%			\item Loops syntax: \tt{for ii in range(N)}
%			\item Create empty list: \tt{my\_list = []}
%			\item Append to list: \tt{my\_list.append(new\_thing)}
%		\end{itemize}
%		}




%%%%%%%%%% Program Criteria %%%%%%%%%
\section*{Program Criteria}
	Write a program that does the following:
	\begin{itemize}
		\item Find all amicable numbers up to some upper bound \tt{N}.
		\item Print out all amicable numbers found in your program.
	\end{itemize}







%%%%%%%%% Deliverables %%%%%%%%%
\section*{Deliverables}
	
	
	Place the following in a folder named \foldername in your repository:
	\begin{itemize}
		\item A Python file \filename  that satisfies the program criteria.
	\end{itemize}

	
\end{document}

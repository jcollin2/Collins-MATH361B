\documentclass{article}
\usepackage{JBpack}
\usepackage{fancyhdr, fancybox}
\usepackage{extramarks, multicol} % Required for headers and footers
\usepackage{sectsty}

\def\prog#1{
\vspace{.1in}\begin{mdframed} \begin{center} \textbf{Programming Reminders} \end{center}#1 \end{mdframed} }



\sectionfont{\underline}




\topmargin=-0.45in
\evensidemargin=0in
\oddsidemargin=0in
\textwidth=7.5in
\textheight=9.0in
\headsep=0.25in 
\headheight = 20pt
\hoffset = -.5in




\pagestyle{fancy}

\lhead{\large Due Date: \duedate}
%\chead{}
\rhead{\large Problem \problemnumber: \probname}
\renewcommand\headrulewidth{0.4pt} % Size of the header rule
\renewcommand{\labelitemii}{$\star$}







%%%%%% Assignment Data %%%%%%%%%%
\newcommand\problemnumber{N.7}
\newcommand\duedate{4/5/2019}
\newcommand\assignmenttype{Number Theory}
\newcommand\foldername{\tt{NumberTheory} }
\newcommand\filename{\tt{N7\_Quadratic\_Name.py}\;\;}
\newcommand\probname{Quadratic Residues}
\newcommand\probnametwo{}







\begin{document}
\begin{multicols}{2}




%%%%%%% Title %%%%%%%%%%	
\begin{minipage}{.5\textwidth}
	 \begin{center} \shadowbox{\begin{Bcenter} \Huge \underline{Problem \problemnumber} \vspace{.25in} \\  \Huge \textbf{\probname} \\ \Huge \textbf{\probnametwo}  \end{Bcenter}  }
	 
	 {\Large Due Date: \duedate
	 
	 Folder: \foldername
	 
	 File Name: \filename
	 
	 Points: 5 points}
	  \end{center}
 \end{minipage}

\columnbreak


%%%%% Learning Objectives %%%%%%%
\begin{minipage}{.45\textwidth}
	\begin{center}
	
	\textbf{\Large Learning Objectives}
	
	\doublebox{
		\begin{Bitemize}
			\item Consider properties of quadratic residues
		\end{Bitemize}
	} 
	\end{center}
\end{minipage}
\end{multicols}






%%%%%%% Background %%%%%%%%%%%%
\section*{Problem Background}
	When considering elements of $ \Zbb_p$, we define a quadratic residue to be the numbers $n \in \Zbb_p$ such that there exists a $k \in \Zbb_p$ where
	\[ n = k^2. \]
	
	For example, let us consider $\Zbb_11$.  Then 1 is an obvious quadratic residue, since $1 = 1^2$.  However, 3 is also a quadratic residue of $\Zbb_11$, since $3 = 6^2 = 36 \equiv 3 \text{(mod)} 11$.  
	
	There are some interesting patterns that occur when we consider $\Zbb_p$ where $p$ is prime.  Therefore, for this problem we assume that $p$ is prime for each of our $\Zbb_p$.  We will consider two questions.  First, how many elements of $\Zbb_p$ are quadratic residues.  The second question we consider is what values for $p$ makes $-1$ a quadratic residue.  Recall that -1 = (p-1) in $\Zbb_p$.  
	
	
 
 	
 	
 	
 	
 	
 	
 	

	
	
	
	
	
	
	
	
%	\prog{
%		\begin{itemize}
%			\item Create function: \tt{def func\_name(params):}
%			\item Loops syntax: \tt{for ii in range(N)}
%			\item Create empty list: \tt{my\_list = []}
%			\item Append to list: \tt{my\_list.append(new\_thing)}
%		\end{itemize}
%		}




%%%%%%%%%% Program Criteria %%%%%%%%%
\section*{Program Criteria}
	Write a program that does the following:
	\begin{itemize}
		\item Create a variable \tt{P} to denote the maximum value for $p$ in your code.
		\item Counts the number of quadratic residues in $\Zbb_p$.
		\begin{itemize}
			\item Create a 2D numpy array \tt{count} to store your results.  This array will have 2 columns:
			\begin{center}
				\begin{tabular}{|c|c|}
				\hline
					$p$ & Number of quadratic residues in $\Zbb_p$	 \\
				\hline
				\end{tabular}
	
			\end{center}
			The first column stores the value for $p$, the second column stores the number of quadratic residues in $\Zbb_p$.	
			\item For all primes $p \leq P$, count the number of quadratic residues in $\Zbb_p$ and store the number in the \tt{count} array.
			\item Print out \tt{count} with appropriate text.
				\end{itemize}
			
			\item Check if $-1 \in \Zbb_p$ is a quadratic residue
			\begin{itemize}
				\item Create a 2D numpy array \tt{neg\_one} to store your results.  This array will have 2 columns:
				\begin{center}
				\begin{tabular}{|c|c|}
				\hline
					$p$ & Is -1 quadratic residue?	 \\
				\hline
				\end{tabular}
				
			\end{center}
			The first column stores the value for $p$, the second column stores a \tt{True} or \tt{False} values, with \tt{True} representing that -1 is a quadratic residue, and \tt{False} stating otherwise.
			\item For all primes $p \leq P$, determine if $-1 \in \Zbb_p$ is a quadratic residue of $\Zbb_p$ and store the result in the \tt{neg\_one} array.
			\item Print out \tt{neg\_one} with appropriate text.
			\end{itemize}
	\end{itemize}







%%%%%%%%% Deliverables %%%%%%%%%
\section*{Deliverables}
	
	
	Place the following in a folder named \foldername in your repository:
	\begin{itemize}
		\item A Python file \filename  that satisfies the program criteria.
		\item A pdf file \tt{N7\_Quadratic\_Name.pdf} created with Latex:
		\begin{itemize}
			\item State any patterns you notice relating the number of quadratic residues in $\Zbb_p$ to the value of $p$.
			\item State any patterns you notices relating whether $-1 \in \Zbb_p$ is a quadratic residue to the value of $p$.
		\end{itemize}
	\end{itemize}

	
\end{document}
